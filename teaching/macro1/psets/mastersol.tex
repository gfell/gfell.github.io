%; whizzy document -advi advi  -html Start-Document

\documentclass[12pt,a4paper]{article}
\usepackage{amsmath}
%\usepackage{advi}
\usepackage{fullpage}
\usepackage{harvard}
\usepackage[dvips]{graphicx}
\usepackage{amsfonts}
\usepackage{amssymb}
\renewcommand{\baselinestretch}{1}
\usepackage{setspace}
\begin{document}
\vspace*{5cm}

\doublespacing \Large \textbf{This document provides the answers to the
questions at least one class teacher was not able to complete during 
class.}
\newpage
\singlespacing

\normalsize

\begin{center}
\textbf{Macroeconomics 1}

Partial solution to problem set 1
\end{center}

\begin{enumerate}
% \item  Please revise the definition of GDP: stress that goods + services are
% produced \textbf{domestically} and \textbf{within the period} of interest.

% \begin{description}
% \item [a)]True, it is a service.

% \item[b)] True it was produced within the period.

% \item[c)] False, the castle was produced in the past. The transaction is just
% a portfolio reallocation. Part of the \underline{stock} of wealth is
% reallocated from physical capital (real estate) to money. Stress the
% difference between stocks (e.g. wealth) and flows (e.g. income). Emphasize
% that a porfolio reallocation does not create any new purchasing power, just
% reallocates existing one across assets.

% \item[d)] False, not produced domestically.
% \end{description}

% \item  Please draw the circular flow diagram and explain why the two sides
% always coincide in value. Stress the accounting convention concerning the
% treatment of inventories (and the storable vs non-storable items distinction).

% Note that in the first table capital income is profits (revenues minus wages
% and cost of raw materials). In the second table, jewelry is the only final
% good (silver is an intermediate good). In the third table, mining uses no
% intermediate input. So its value added coincides with the full value of its
% product. Jewelry uses 300(K) \pounds\ of silver as an intermediate input. So
% its value added is only 700.%

% \begin{tabular}
% [c]{c|c|cc}%
% & \multicolumn{2}{c}{Income side} & \\\hline
% \multicolumn{1}{|c|}{} & Mining & Jewelry & \multicolumn{1}{|c|}{Total}%
% \\\hline
% \multicolumn{1}{|c|}{Labour income} & 200(*) & 250(*) &
% \multicolumn{1}{|c|}{450}\\\hline
% \multicolumn{1}{|c|}{Capital income} & 100(*) & 450(*) &
% \multicolumn{1}{|c|}{550}\\\hline
% \multicolumn{1}{|c|}{Total} & 300 & 700 & \multicolumn{1}{|c|}{1000}\\\hline
% \end{tabular}
% $%
% \begin{tabular}
% [c]{|c|c|cc}%
% \multicolumn{3}{c}{Output side (production of final goods)} & \\\hline
% &  Mining & Jewelry & \multicolumn{1}{|c|}{Total}\\\hline
% $\text{Value of final goods}$ & 0(*) & 1000(*) & \multicolumn{1}{|c|}{1000}%
% \\\hline
% \end{tabular}
% \ \ \ $%

% \begin{tabular}
% [c]{|c|c|cc}%
% \multicolumn{3}{c}{Output side (value added)} & \\\hline
% &  Mining & Jewelry & \multicolumn{1}{|c|}{Total}\\\hline
% $\text{Value }$added & 300(*) & 700(*) & \multicolumn{1}{|c|}{1000}\\\hline
% \end{tabular}

% \item [3.] Please explain aggregation problem (cannot sum apples and oranges).
% Everything needs to be reduced to a common unit of account. That's why we need
% to use prices to aggregate. Define nominal (current prices) and real (base
% year prices)\ GDP. Conduct the analysis in general terms rather than giving
% them a number initially (put the numbers on the board just at the end of the exercise).

% Stress that nominal and real GDP coincide (trivially by definition) in the
% base year.

% Note that the quantity of apples in 1998 $Q_{a}^{98}$ is zero. So the ratio
% between real GDP at 1998 prices is given by
% \begin{equation}
% \frac{P_{a}^{98}Q_{a}^{99}+P_{c}^{98}Q_{c}^{99}}{P_{c}^{98}Q_{c}^{98}}%
% =\frac{P_{a}^{98}}{P_{c}^{98}}\frac{Q_{a}^{99}}{Q_{c}^{99}}+\frac{Q_{c}^{99}%
% }{Q_{c}^{98}}.
% \end{equation}
% The ratio when real GDP\ is measured in 99 prices is%
% \begin{equation}
% \frac{P_{a}^{99}Q_{a}^{99}+P_{c}^{99}Q_{c}^{99}}{P_{c}^{99}Q_{c}^{98}}%
% =\frac{P_{a}^{99}}{P_{c}^{99}}\frac{Q_{a}^{99}}{Q_{c}^{99}}+\frac{Q_{c}^{99}%
% }{Q_{c}^{98}}.
% \end{equation}
% So the change in \textbf{real} purchasing prices depens on \textbf{relative
% }prices in the base year. This is undesirable.

% \begin{description}
% \item [a)]What is nominal GDP in 1998? \textbf{90,000} And in 1999? \textbf{100,020}

% \item[b)] Using 1998 as the base year (i.e. using 1998 prices) what is real
% GDP in 1998? \textbf{90,000} And in 1999? \textbf{75,040}

% \item  What is the ratio between real GDP in 1999 and 1998? \textbf{75,040/90,000}

% \item[c)] Using 1999 as the base year (i.e. using 1999 prices) what is real
% GDP in 1998? \textbf{120,000} And in 1999? \textbf{100,020}

% \item  What is the ratio between real GDP in 1999 and 1998? \textbf{100,020/120,000}
% {\hspace*{2cm}}
% \end{description}

% The two numbers differ for the reason discussed above (relative prices were different).

\item [4.] Suppose Toyota buys BMW German plants. The take-over has no effect on
the quantity of car produced at the plants.

What is the effect of the acquisition on the following variables in subsequent
years (choose between \emph{up}, \emph{down} and \emph{unchanged})?

\begin{description}
\item [a)]German GDP \underline{\textbf{unchanged}} the cars are still
produced domestically

\item[b)] German GNP \textbf{falls }the profits now accrue to Japanese
shareholders who are not resident in Germany.\newpage
\end{description}
\end{enumerate}

\begin{center}
\textbf{Macroeconomics 1}

Partial solutions to problem set 2
\end{center}

% Stress that all components of GDP are \textbf{flows of goods and
%   services; }i.e. neither transfers nor portfolio reallocation.

 \begin{enumerate}

% \item $C$ will increase if individuals

%   \begin{description}
%   \item [a)]true

%   \item[b)] false, portfolio reallocation

%   \item[c)] true, it is an increase in transfers to private agents. It
%     increases disposable income and induces them to buy more
%     \textbf{goods and services}
%   \end{description}

% \item $I$ will fall if

%   \begin{description}
%   \item [a)]false, portfolio reallocation

%   \item[b)] false, no increase in goods + services produced

%   \item[c)] false, see a)

%   \item[d)] false, it will increase
%   \end{description}

% \item Please write down the goods market equilibrium condition. Stress
%   that it requires \textbf{desired} expenditure $Z$ to equal
%   production. So it is a situation in which there is no undesired
%   accumulation of inventories. This is what makes it an
%   \textbf{equilibrium condition} rather than an \textbf{accounting
%     identity. }The circular flow identity implies that production
%   equals \textbf{actual }expenditure (that is expenditure including
%   any undesired accumulation of inventories). It is an identity
%   because it always holds (not only at equilibrium).

%   Write down the system
%   \begin{align}
%     Z  &  =C+I+G\\
%     Y & =Z
%   \end{align}
%   \textbf{Stress} that it is a system of two equations in two unknowns
%   (Please stress over and over that we always need as many equations
%   and endogenous variables). So, it is determined. Rederive the
%   Keynesian cross.

%   \begin{description}
%   \item [a)]Write down the goods market equilibrium condition?
%     $\mathbf{Y=C+I+G}%
%     .$ What is the value of equilibrium output? \textbf{800}

%   \item[b)] Suppose the government increases spending by 100. In the
%     new equilibrium, what are the values of:

%     \begin{description}
%     \item [i)]income? \textbf{1200}

%     \item[ii)] demand? \textbf{1200}

%     \item[iii)] consumption? \textbf{925}
%     \end{description}
%   \end{description}
\item [4.] If taxes increase with income, disposable income responds less
  than one to one to changes in gross income. This is because taxes
  fall in recessions (when $Y$ is low) and increase in booms. So,
  consumption and equilibrium output fluctuate less in response to
  shocks that affect equilibrium income.

  \begin{description}
  \item [a)]Write down the expression for equilibrium output. $Y=\frac
    {1}{1-c_{1}(1-t)}\bar{Z}$ with
    $\bar{Z}=\bar{C}-c\bar{T}+\bar{I}+\bar{G}.$

  \item[b)] The Keynesian multiplier is $\frac{1}{1-c_{1}(1-t)}$. With
    respect to the case discussed in the lecture (constant taxes),
    does the economy respond more or less to a change in autonomous
    spending?\ It responds less.  The Keynesian multiplier gives the
    change in equilibrium output with respect to a change in
    autonomous spending $\bar{Z}.$
    \begin{equation}
      \frac{\Delta Y}{\Delta\bar{Z}}%
      % TCIMACRO{\QATOPD{|}{.}{{}}{_{\text{proportional tax}}}}%
      % BeginExpansion
      \genfrac{|}{.}{0pt}{}{{}}{_{\text{proportional tax}}}%
      % EndExpansion
      =\frac{1}{1-c_{1}(1-t)}<\frac{1}{1-c_{1}}=\frac{\Delta
        Y}{\Delta\bar{Z}}%
      % TCIMACRO{\QATOPD{|}{.}{{}}{_{\text{lump-sum tax}}}}%
      % BeginExpansion
      \genfrac{|}{.}{0pt}{}{{}}{_{\text{lump-sum tax}}}%
      % EndExpansion
    \end{equation}
    The Keynesian multiplier is smaller with proportional taxes (for
    the reason mentioned above). That is why direct taxes (which in
    the real world are usually progressive) are called and automatic
    stabilizer.
  \end{description}

\end{enumerate}
\newpage
\begin{center}
  \textbf{Macroeconomics 1}

  Partial solutions to problem set 3
\end{center}

\begin{enumerate}

% \item  Use the IS curve equation to show that with investment dependent on the
% interest rate, the Keynesian multiplier now gives the horizontal shift in the
% IS\ curve. Conduct the analysis both using the equation and graphically.

% \begin{description}
% \item [a)]True, think of it as a change in $c_{0}.$ Show that it is an IS
% curve shift.

% \item[b)] false, it is a movement \emph{along} a given IS curve.

% \item[c)] True, same as a).

% \item[d)] false, it is a portfolio reallocation and does not affect goods
% market equilibrium.
% \end{description}

% \item To save time, just set to zero the relevant coefficients in your
%   previous derivation of the IS curve. With $I=a$ and $C=c_{0}$ it is
%   \begin{equation}
% \label{eq:2}
%     Y=  C+I+\bar G =c_{0}+a+\bar{G}.
%   \end{equation}


%   \begin{description}
%   \item [a)]False. IS curve definition: locus of output/interest rate
%     combinations for which the goods market is in equilibrium. This is
%     the case for \emph{any }level of the interest rate and output given
%     by equation \eqref{eq:2} above. So \eqref{eq:2} \emph{is} an IS
%     curve.

%   \item[b)] True.

%   \item[c)] False it is vertical.

%   \item[d)] True, equilibrium income is independent from the interest
%     rate.
%   \end{description}
\setcounter{enumi}{2}
\item Please derive labour market equilibrium. Explain that equation 1
  (the price setting equation) comes from profit maximization (marginal
  revenue=marginal cost), identifying MC and reminding students that
  unless there is perfect competition the price level is set as a mark
  up over marginal revenue and cost. Justify equation 2 (the wage
  setting curve) stressing that workers wage claims are likely to be
  lower when unemployment is higher and when outside sources of income
  (e.g. unemployment benefits) are lower. Stress that the model assumes
  flexible wages and prices and correct expectations for both workers
  and firms (long-run assumptions). Derive equilibrium both graphically
  in the $\left( W/P,u\right) $ space and algebraically.  Stress that we
  have two equations and three endogenous variables.  Yet, since two of
  them enter only as a ratio (the real wage) we can determine the real
  wage and the unemployment rate. Emphasize that the only variables that
  affect equilibrium are the price mark up $\mu$ and $z.$ Use the
  production function to derive the long run labour market equilibrium
  locus $Y=\bar Y\left( \mu,\bar{z},\bar L\right) $ in the $\left(
    Y,i\right) $ space.

  \begin{enumerate}
  \item [a)]$W/P=1/(1+\mu)=1/1.05$, $Y=N=895$ and $u=10.5\%$.

  \item[b)] Show graphically how the price setting curve shifts down
    (explain the intuition, firms want to make higher profits so the
    real wage has to fall) and how the equilibrium real wage falls and
    $u$ goes up as workers are willing to work at the lower real wage
    only if the cost of being unemployed (negatively related to $u$)
    increases. Show how the MRLE locus shifts left.

    i) The real wage decreases

    ii) Output decreases

    iii) The unemployment rate increases

  \item[c)] Suppose that $\bar{z}$ falls to 0.05. Show that the wage
    setting curve shifts down as workers wage demands are lower. The
    real wage is unaffected since it is determined by the price mark
    up, but now more workers are willing to work at that wage and the
    unemployment rate falls.

    i) The real wage is unchanged

    ii) Output increases

    iii) The unemployment rate decreases
  \end{enumerate}

\item In the long run labour market equilibrium determines the real
  wage.  So if the nominal wage halves prices have to halve to keep the
  real wage unchanged (given that firms want to keep a constant mark up
  of prices over costs). Stress why nominal variables do not matter (no
  money illusion: workers are interested in the real purchasing power of
  their wage and firms in the mark up).

  \begin{enumerate}
  \item [a)]false

  \item[b)] false

  \item[c)] false
  \end{enumerate}

\end{enumerate}
 \newpage
 \begin{center}
  \textbf{Macroeconomics 1}

  Partial solution to problem set 4
\end{center}



\begin{enumerate}
% % \item Here the interest rate is the endogenous variable that clears
% %   the money market. Draw the demand and supply for \textbf{nominal
% %     money }in the interest-nominal M space. Note that nominal money
% %   demand is $M^{d}=PYL(i).$ Show which shock shifts which curve and
% %   discuss how the interest rate moves to reestablish equilibrium.

% %   \begin{enumerate}
% %   \item [a)]True, the interest rate rises to reduce money demand until
% %     equilibrium is reestablished.

% %   \item[b)] False, it reduces nominal money demand at any level of $i.$
% %     Given that the nominal money supply is unchanged, $i$ has to fall
% %     to bring money demand back to its original level.

% %   \item[c)] False, it increases nominal income hence nominal money
% %     demand. $i$ has to rise to keep nominal money demand in line with
% %     the unchanged supply.

% %   \item[d)] False, it reduces it. It follows from either b) or c).
% %   \end{enumerate}


% % \item Derive the LM curve as the combinations of output and interest
% %   rate levels such that the money market is in equilibrium for a given
% %   price level (FYOI: this last qualifier is there to ensure that the
% %   LM curve shifts with changes in the price level even in the medium
% %   run).

% %   \begin{enumerate}
% %   \item [a)]True, $M$ increases. The LM shifts right as the
% %     equilibrium interest rate has to fall at any level of output to
% %     bring nominal money demand in line with the higher supply.

% %   \item[b)] Suppose money demand takes the form $M^{d}=PY(a-i)$ where
% %     $a$ is a positive parameter. The shock can be represented by an
% %     increase in $a.$ At unchanged $Y$ money demand is higher. Since
% %     the nominal money supply is given the interest rate has to
% %     increase to bring money demand back in line with supply. Hence,
% %     $LM$ shifts up.

% %     At this point answer d) before c).

% %   \item[d)] This is a movement along a given LM curve.

% %   \item[c)] This is tricky. It depends on whether the change in
% %     nominal income is due to a change in real income at unchanged
% %     prices (same as e) ) or if prices change. Only in the latter case
% %     the LM shifts.
% %   \end{enumerate}
% \item Stress that which variable/s clears the money market depends on
%   the model. For example, in the long run we have seen that output
%   and the interest rate are determined on the labour and goods market.
%   Hence, they cannot adjust to clear the money market. So, prices have
%   to do the job. Note that if you draw nominal money demand and supply
%   in the interest-nominal M space, now the change in the price level
%   shifts the $M^{d}$ curve until it intersect the money supply curve
%   at the exogenously given $i.$ Show which shock shifts which curve
%   and discuss how the price level moves to reestablish equilibrium.

%   \begin{enumerate}
%   \item [a)]Falls, the price level has to fall to reduce the nominal
%     demand for money and reestablish equilibrium.

%   \item[b)] True, the fall in real income reduces money demand. The
%     price level has to rise to keep demand in line with the unchanged
%     supply.

%   \item[c)] False, it increases it. Now equilibrium has to take place
%     at a higher level of the exogenous interest rate. At this higher
%     $i$ if prices are unchanged nominal money demand is below supply.
%     Hence the price level has to rise to increase money demand.

%   \item[d)] False, nominal income is an endogenous variable (since it
%     depends on $P$ that is endogenous). So, for given interest rate
%     and nominal money supply nominal income is fully pinned down to
%     the level that ensures that the money market clears and cannot
%     change. Any fall in real income will be offset by an incrase in
%     $P.$
%   \end{enumerate}

\item [4.] Now the interest rate is fixed at the level desired by the central
  bank.  In order to keep the interest rate fixed, the central bank has
  to supply any amount of money that the public demands at that interest
  rate. That is, as output and money demand change, the central bank
  adjusts the supply of money to demand. This means that it loses
  control of the money supply. As in a standard demand/supply framework
  if you control the supply of one commodity the price adjusts to clear
  the market, but if you want to fix the price then you have to adjust
  supply in response to any change in demand at the fixed price. The LM
  curve is then horizontal at the target interest rate.

\end{enumerate}
  \newpage
\begin{center}
  \textbf{Macroeconomics 1}

Partial  solution to problem set 5
\end{center}

% % Stress that we are talking long run equilibrium here. Just remind them
% % that in the long run labour market equilibrium fully determines the
% % unemployment rate, employment and the level of output; i.e. the LRLE
% % locus is vertical. Since now output is determined on factor markets
% % it cannot respond to demand. Viceversa it is demand which has to adjust
% % to equal production and maintain goods market equilibrium.  Derive the
% % general equilibrium of this economy in the output interest rate
% % space. Shift in the IS curve can only affect the interest rate and the
% % composition of aggregate demand but not its level. The price level has
% % to adjust (LM\ has to shift) to ensure that the money market clears.

% % Derive the AD curve from the IS and LM. Explain it denotes equilibrium
% % on \textbf{both} the goods and money market in the $(Y,P)$ space.

 \begin{enumerate}
% % \item The price setting curve shifts down, hence equilibrium
% %   unemployment increase and equilibrium output falls.

% %   In the $(Y,i)$ space the LRLE curve shifts to the left, the IS
% %   curve is unchanged. In the $(Y,P)$ space the MRAS shifts left and the
% %   AD curve is unaffected.

% %   \begin{enumerate}
% %   \item [a)]False. Disposable income falls.

% %   \item[b)] True. As the marginal propensity to consume is smaller
% %     than 1, consumption falls less than output. So, investment also
% %     has to fall to reestablish goods market equilibrium.

% %   \item[c)] False. The interest rate has to rise for equilibrium
% %     investment to fall.

% %   \item[d)] True, the price level increases, for real balances to fall
% %     in line with lower real money demand.
% %   \end{enumerate}

% % \item In the $(Y,i)$ space the IS curve shifts left, the LRLE
% %   curve is unchanged. In the $(Y,P)$ space the AD curve shifts left and
% %   the LRAS curve is unaffected.


% %   \begin{enumerate}
% %   \item [a)]False. Output is determined on the labour market.

% %   \item[b)] False, it increases by 100. Investment is the only
% %     component of expenditure that can change to reestablish goods
% %     market equilibrium.

% %   \item[c)] True, this is the only way that investment can increase by
% %     100.

% %   \item[d)] True, the price level falls, for real balances to increase
% %     in line with higher real money demand.
% %   \end{enumerate}


% % \item In the $(Y,r)$ space the IS curve shifts right, the LRLE  curve is
% %   unchanged. In the $(Y,P)$ space the AD curve shifts right and the LRAS
% %   curve is unaffected.

% %   \begin{enumerate}
% %   \item [a)]False. Unchanged.

% %   \item[b)] False. As the equilibrium interest rate goes up. Remind
% %     them of the equivalence between goods market equilibrium and
% %     equilibrium on the market for loanable funds.

% %   \item[c)] True.

% %   \item[d)] False, the price level increases, for real balances to
% %     fall in line with lower real money demand.
% %   \end{enumerate}
\setcounter{enumi}{3}
\item The government increases government expenditure $G$ by 200. If
  the consumption function is given by%
  \begin{equation}
    C=100+0.7(Y-\bar{T})
  \end{equation}
  and the investment function is the same as in equation 1.

  Curves shift as in the previous question.  Discuss that a given amount
  of expenditure can be financed through taxes or debt. The latter way
  is more expansionary (i.e. we are assuming Ricardian equivalence does
  not hold) at given interest rate, so it requires a bigger fall in
  investment to keep the goods market in equilibrium.

  In both cases the long run equilibrium level of output cannot
  change as it is determined on the labour market. So total aggregate
  demand cannot change.

  \begin{enumerate}

  \item[a)] $\Delta\bar{G}=200,$ $\Delta\bar{T}=0.$

    \begin{enumerate}
    \item [i)]$\Delta G=200,$ $\Delta C=0.$ So we require $\Delta
      I=-200$ which calls for $\Delta i=0.2$ (a bigger increase in the
      the long run equilibrium interest rate)

    \item[ii)] the change in the long run equilibrium level of
      consumption $\Delta C=0$

    \item[iii)] the change in the long run equilibrium level of
      aggregate saving $\Delta S=\Delta S_{G}=\Delta I=-200.$
    \end{enumerate}

  \item [b)]$\Delta\bar{G}=\Delta\bar{T}=200$

    \begin{enumerate}
    \item [i)]$\Delta G=200,$ $\Delta C=-140.$ So we require $\Delta
      I=-60$ which calls for $\Delta i=0.06$ (an increase in the the
      long run equilibrium interest rate)

    \item[ii)] the change in the long run equilibrium level of
      consumption $\Delta C=-140$

    \item[iii)] the change in the long run equilibrium level of
      aggregate saving $\Delta S=S_{p}=\Delta I=-60.$ Government
      saving is unaffected (balanced budget), but since a balanced
      budget fiscal expansion
    \end{enumerate}
  \end{enumerate}
\end{enumerate}

% \newpage
% \begin{center}
%   \textbf{Macroeconomics 1}

% Partial  solution to problem set 6
% \end{center}

% % Stress that we are talking long run equilibrium here.

% % Please write down the system of equations that describes general
% % equilibrium (IS, LM\ and LRLE). Show the corresponding graphical
% % devices (AD\ and LRAS) in the output/price space. Emphasize (by
% % deriving it graphically) that the aggregate demand curve describes
% % equilibrium on \emph{both }the goods \emph{and} money markets (it is
% % in fact a reduced form of the IS and LM). For each change discuss
% % which curve shifts in response to the shock in each graph and then
% % derive the process to adjustment to the new equilibrium. Emphasize
% % that demand cannot affect output (determined on the labour market in
% % the long run). Fiscal policy can affect the composition of demand
% % and the interest rate, but monetary policy only nominal variables ($P$
% % and the nominal wage $W$).

% \begin{enumerate}
% % \item If the government reduces expenditure. The IS\ and AD shift left
% %   (why?\ explain the economic intuition).

% %   \begin{enumerate}
% %   \item [a)]False, nothing happens on the labour market.

% %   \item[b)] True.

% %   \item[c)] False, as demand falls but output not the interest rate
% %     has to fall for investment to rise.

% %   \item[d)] False, as the interest rate falls, real money demand
% %     (measured in units of output) increases. The price level has fall
% %     to increase the real supply of money and reestablish money market
% %     equilibrium.
% %   \end{enumerate}

% % \item If consumers want to save less. Model it as a rise in $c_{0}.$
% %   The IS\ and AD curves shift rightt. The equilibrium interest rate
% %   and price rise.

% %   \begin{enumerate}
% %   \item [a)]False, exactly because output (and disposable income) is
% %     the same but the level of consumption is higher, saving falls.

% %   \item[b)] False, with higher consumption investment has to fall to
% %     keep total desired expenditure unchanged. In fact, the interest
% %     rate rises to achieve this.

% %   \item[c)] False, see point b).

% %   \item[d)] False, this is just the reverse of point d) in question 1
% %     above.
% %   \end{enumerate}

% % \item For any level of $Y,$ $r$ the demand for real balances $L=YL(r)$
% %   must fall. If you take a simple form for $L(r)=a-r$ this is a fall
% %   in the intercept $a.$ This is shift to the right in the LM\ and AD
% %   curves (at unchanged $P$ and $r,$ $Y$ has to increase for the real
% %   money demand to stay in line with supply). In equilibrium $Y$ and
% %   $r$ cannot change. So, equilibrium real money demand is lower. The
% %   price level increases to produce the necessary fall in real
% %   balances. Hence, the LM\ curve shifts all the way back to its
% %   original position.

% %   \begin{enumerate}
% %   \item [a)]False, the price level adjusts to generate the necessary
% %     increase in \emph{real }balances.

% %   \item[b)] False, determined on the goods and labour markets alone.

% %   \item[c)] False, it will increase.

% %   \item[d)] False, determined on the labour market alone.
% %   \end{enumerate}
% \setcounter{enumi}{3}
% \item Consider an economy in which the long run equilibrium level of
%   output is given by $\bar Y=2000$ and in which the IS and LM loci
%   \begin{align}
%     IS\text{ }Y  &  =2200-1000r\\
%     LM\text{ }\frac{\bar{M}}{P} & =Y\left( 1-r\right)
%   \end{align}
%   describe goods and asset markets equilibrium.

%   \begin{enumerate}
%   \item If the nominal money supply is $\bar{M}=800$ what are the
%     equilibrium values of output \underline{2000}, the interest
%     rate\underline{ 0.2} and the price level \underline{0.5}.

%     Output is determined on the labour market and unchanged. To solve
%     the real interest replace for output in the IS curve to obtain 

% \[ 
% 2000 = 2200-1000r
% \]
%  whose solution is $r=0.2.$ To solve for the price level replace for
%  output and the real interest rate in the LM curve to obtain

% \[\frac{800}{P}=2000\left(1-0.2\right),\]
% whose solution is $P=0.5.$

%    \item Please use the graphs derived above shifting the relevant
%     curves and deriving the new equilibrium graphically. The nominal
%     money supply increases by 50\%. With $Y$ and $r$ unchanged the
%     real demand for money is unchanged and so has to be the real money
%     supply in equilibrium. So, the only change is a 50\% increase in
%     the price level which leaves real balances unchanged.

%     If the nominal money supply increases to $\bar{M}=1200$ what are
%     the new equilibrium values of output\underline{ 2000}, the
%     interest rate \underline{0.2} and the price level\underline{
%       0.75}.
%   \end{enumerate}
% \end{enumerate}

% % \newpage
% % \begin{center}
% %   \textbf{Macroeconomics 1}

% %   Solution to problem set 7
% % \end{center}
% % Spot test held. 
% % \newpage
% % \begin{center}
% %   \textbf{Macroeconomics 1}

% %   Solutions to problem set 8
% % \end{center}

% % \begin{enumerate}
% % \item Quickly derive the LRAS from labour market equilibrium and remind
% %   students why the SRAS is horizontal (firms are off the PS curve, they
% %   are not maximizing profits in the short run, but still making positive
% %   profits) and how it shifts over time as firms adjust their price
% %   towards the long run value.

% %   \begin{enumerate}
% %   \item [a)]An exogenous increase in the price level shifts the LRAS
% %     (\emph{false, independent from price}) and the SRAS (\emph{true it
% %       shifts it up.})

% %   \item[b)] An increase in the price mark up shifts the LRAS \emph{true,
% %       it shifts left. If firms pay a
% %       lower real wage unemployment has to go up to convince workers to
% %       accept it.} and the SRAS \emph{false, not on impact.}

% %   \item[c)] An increase in the minimum wage shifts the LRAS and the SRAS
% %     (\emph{can be modelled as an increase in $\bar z.$ Same as b). Now
% %       the reason why employment has to fall is to reconcile workers wage
% %       demands with firms' willingness to pay}).

% %   \item[d)] An increase in the nominal wage shifts the LRAS
% %     (\emph{false it has no effect}) and the SRAS(\emph{false firms
% %       supply any amount of output at $\bar P$ independently from $W.$})
% %   \end{enumerate}
% % \end{enumerate}

% % In what follows assume that before the shock all markets are both in
% % short run and long run equilibrium. The questions ask you to derive the
% % new short run equilibrium after the shock.

% % Write down the short run general equilibrium equations in both sets of
% % axes.\ The SRAS-AD diagram shows the equilibrium price level. Stress
% % that with $\bar{P},$ the labour market determines $P$ and the IS and LM
% % curves (together with $P$ determined on the labour market) determine
% % output and the real interest rate.  Keep always the LRLE, LRAS loci on
% % the graphs just as a reference.

% % \begin{enumerate}
% % \item [2.]IS\ and AD shift right. The interest rate increases but also
% %   equilibrium output. Crowding out of investment is only partial
% %   contrary to long run. The economic intuition is the following. The
% %   higher demand for consumption increases output through the
% %   multiplier. Firms produce more and employ more workers (nominal and
% %   real wages increase). The increase in output raises real money demand,
% %   the interest rate has to increase to reestablish money market equilibrium.

% %   \begin{enumerate}
% %   \item [a)]The short run equilibrium level of output increases. True.

% %   \item[b)] The short run equilibrium interest rate will fall. False,
% %     it increases.

% %   \item[c)] The short run equilibrium level of investment increases.
% %     False, it falls.

% %   \item[d)] The short run equilibrium price level will increase. True.
% %   \end{enumerate}

% % \item[3.] If the central bank increases the supply of base money. The
% %   interest rate falls to clear the money market, which increases
% %   output through higher investment. Firms increase production in
% %   response to the increase in demand driving nominal wages and prices
% %   up.

% %   \begin{enumerate}
% %   \item [a)]The short run equilibrium level of output increases. True.

% %   \item[b)] The short run equilibrium interest rate will fall. True.

% %   \item[c)] The short run equilibrium level of investment increases.
% %     True.

% %   \item[d)] The short run equilibrium price level will increase. True.
% %   \end{enumerate}

% % \item[4.] If the government reduces government expenditure. This is
% %   just the opposite of the fall in saving above. So everything we said
% %   about it applies (reversing the direction of the changes).

% %   \begin{enumerate}
% %   \item [a)]The short run equilibrium level of output increases.
% %     False.

% %   \item[b)] The short run equilibrium interest rate will fall. True.

% %   \item[c)] The short run equilibrium level of investment increases.
% %     True.

% %   \item[d)] The short run equilibrium price level will increase.
% %     False.
% %   \end{enumerate}

% % \item[5.] If the government reduces government expenditure by the same
% %   amount as in question 4 and reduces taxes by an equal amount. A
% %   balanced-budget reduction in expenditure is still (though less)
% %   contractionary. Show students that the change in the autonomous
% %   component of expenditure is s  maller.

% %   \begin{enumerate}
% %   \item [a)]False, it false but by less than in question 4.

% %   \item[b)] The short run equilibrium interest rate falls by less than
% %     in question 4.

% %   \item[c)] The short run equilibrium price level falls by less than
% %     in question 4.
% %   \end{enumerate}
% % \end{enumerate}
%  \newpage
% \begin{center}
%   \textbf{Macroeconomics 1}

% Partial  solution to problem set 9
% \end{center}

% Please write down the short run general equilibrium equations in both
% sets of axes.  Keep always the LRLE, LRAS loci on the graphs just as a
% reference.

% \begin{enumerate}
% \item Fischer equation implies that $i=r+\pi^{e}.$ So changes in
%   inflationary expectations shift the LM curve alone in the
%   output/\emph{real }interest rate space. LM shifts down (money demand
%   falls at unchanged $r$ as $i$ increases). So at unchanged real
%   interest rate and prices output needs to increase to reestablish money
%   market equilibrium. AD shifts right. \textbf{Do the long run
%     first. }

%   \begin{enumerate}
%   \item [a)]Real interest rate decreases. Prices are unchanged. So the LM
%     curve does not shift all the way back and the real interest rate has
%     to fall to clear the money market.  The fall in the real interest
%     rate boosts investment and output. Since output is higher but the
%     real money supply is unchanged the nominal interest rate has to
%     increase to keep the real money demand unchanged in the face of
%     higher output. It follows that $\Delta i=\Delta r +\Delta \pi^{e}>0.$ 

%   \item[b)] Real interest rate is unaffected as neither the IS nor the
%     MRLE\ have been affected (classical dichotomy holds and expected
%     inflation is a nominal variable). The LM curve shifts all the way
%     back. The nominal interest rate increases one for one with expected
%     inflation ($\Delta i=\Delta \pi^{e}$). Output is unchanged and the
%     price level needs to increase to bring the supply of real balances
%     in line with lower demand. Note the crucial role of price adjustment
%     to clear the money market without changes in the real interest rate.
%   \end{enumerate}

% % \item IS and AD shift left. The interest rate falls but also equilibrium
% %   output. So unlike in the long run, the net effect on private and
% %   aggregate saving and investment is ambiguous. This is called the
% %   paradox of thrift (attempts by agents to save more may actually result
% %   in lower investment and saving in the short run). The economic
% %   intuition is the following. The lower demand for consumption reduces
% %   output through the multiplier. Firms produce less and employ less
% %   workers. The fall in output reduces money demand and the interest
% %   rate. The net effect on investment, hence on aggregate saving, is
% %   ambiguous. Stress that the ambiguity is not there if investment is
% %   independent of output as we have assumed throughout the course or in
% %   the long run when equilibrium output does not change.

% % \begin{enumerate}
% % \item [a)]The short run equilibrium level of output increases. False.

% % \item[b)] The short run equilibrium interest rate will fall. True.

% % \item[c)] The short run equilibrium level of investment increases. Uncertain
% % as both output and the interest rate fall.

% % \item[d)] The short run equilibrium price level will increase. False, unchanged.
% % \end{enumerate}
% % \item For investment to fall at unchanged output level the interest
% %   rate has to increase. So the necessary policy mix is one which
% %   increases the interest rate and leaves output unchanged. This
% %   requires an expansionary fiscal policy, which increases the interest
% %   rate but in the short run increases output, coupled with a
% %   contractionary monetary policy to keep output at its long run
% %   equilibrium level. The IS curve shifts right and the LM curve left
% %   in such a way that it intersects the new IS curve where the latter
% %   crosses the MRLE.  Since output is unchanged, the AD curve does not
% %   shift and the price level is unchanged both in the short and long
% %   run.

% %   \begin{enumerate}
% %   \item [a)]Expansionary fiscal and monetary policy. False, it would
% %     boost output in the short run.

% %   \item[b)] Contractionary fiscal policy and expansionary monetary
% %     policy. False

% %   \item[c)] None of the above. True
% %   \end{enumerate}
% \setcounter{enumi}{3}
% \item Please derive the money supply multiplier $mm=M/H$ using the
%   definition of $Cu=cM,\ D=\left( 1-c\right) M$ and $R=\theta D.$ We
%   can write $H=CU+R=cM+\theta\left( 1-c\right) M.$ $mm=1/\left(
%     c+\theta\left( 1-c\right) \right) =1/\theta$ given that $c=0$ by
%   assumption. So, all $H$ takes the form of reserves and all notes and
%   coins are owned by banks (which is trivial given that private agents
%   do not hold any currency).

%   Money supply multiplier in this economy? $mm=1/\theta=1/0.25=4.$ If
%   the supply of base money is \pounds100, what is the supply of
%   $M_{1}?$ $M=mm\cdot H=4\cdot100=400.$ In equilibrium, who holds notes
%   and coins in this economy? Banks.
%  \end{enumerate}
% % \newpage
% % \begin{enumerate}
% % \item Write down the relationship between $M_{1}$ and base money
% %   $M=mm\cdot H,$ with $mm=1/\left( c+\theta\left( 1-c\right) \right)
% %   .$ $mm$ affects LM through $M.$ Write down and draw all curves
% %   involved.

% %   \begin{enumerate}
% %   \item [a)]$H$ increases, hence $M$ increases. LM and AD shift
% %     rightward. $Y$ increases, $i$ falls and $P$ increases.

% %   \item[b)] $mm$ is decreasing in $c.$ So, $M$ falls for given $H.$
% %     Opposite of a).

% %   \item[c)] $mm$ is decreasing in $\theta.$ Same as b).

% %   \item[d)] The question makes no sense (but it serves a purpose!).
% %     Nominal income is endogenous, so it cannot change unless some
% %     exogenous variable changes.

% %   \item[e)] Same as d).
% %   \end{enumerate}

% % \item From the wealth identity $W=B^{d}+H^{d}$ as only base money
% %   constitutes net wealth for the private sector. Hence,
% %   $B^{d}=W-H^{d}.$ $H^{d}=M^{d}/mm.$ Hence%
% %   \begin{equation}
% %     H^{d}=PY(.35-i)
% %   \end{equation}
% %   where $PY$ is nominal income and $i$ the nominal interest rate.
% %   \begin{equation}
% %     B^{d}=W-H^{d}=W-PY(.35-i). \label{b}%
% %   \end{equation}
% %   Plugging in numbers we obtain $B^{d}=60,000-50,000(.35-i).$

% %   By how much would the demand for bond change in response to an
% %   increase in the interest by 20 percentage points? $\Delta
% %   B^{d}=50,000\Delta i=10,000$ (it would increase by 10,000).

% %   Suppose wealth falls and nominal income and the interest rate are
% %   unchanged.  Tick the correct alternative.

% %   \begin{enumerate}
% %   \item [a)]False, money demand does not depend on wealth.

% %   \item[b)] Bond demand falls (see equation (\ref{b}) ).

% %   \item[c)] False, since the money supply is unchanged and money
% %     demand is unchanged the LM\ curve is unaffected.
% %   \end{enumerate}

% %   Suppose nominal income falls and wealth and the interest rate are
% %   unchanged.  Tick the correct alternative.

% %   \begin{enumerate}
% %   \item [a)]False, it falls of course.

% %   \item[b)] True, as the volume of nominal transactions falls people
% %     want to hold less money. Hence for given wealth they demand more
% %     bonds as it is apparent from equation (\ref{b}).\newpage
% %   \end{enumerate}
% % \end{enumerate}

% % \begin{center}
% %   \textbf{Macroeconomics 1}

% %   Solutions to problem set 10
% % \end{center}

% % \begin{enumerate}
% % \item The IS curve shifts left, the LM curve right, the AD may shift
% %   right or left depending on whether the contractionary fiscal policy
% %   is more or less than offset by the expansionary monetary policy. So,
% %   the interest rate falls both in the short and long run. Output may
% %   increase or fall in the short run, but is unchanged (determined on
% %   the labour market) in the long run. The price level may increase
% %   (if the AD shifts right) or fall (if the AD shifts left) both in the
% %   short and long run.

% %   \begin{enumerate}
% %   \item [a)]The short run equilibrium level of output increases.
% %     Uncertain.

% %   \item[b)] The short run equilibrium interest rate falls. True.

% %   \item[c)] The short run equilibrium price level increases.
% %     Uncertain.

% %   \item[d)] The long run equilibrium level of output increases.
% %     False.

% %   \item[e)] The long run equilibrium interest rate falls. True.

% %   \item[f)] The long run equilibrium price level increases.
% %     Uncertain.
% %   \end{enumerate}

% % \item For investment to fall at unchanged output level the interest
% %   rate has to increase. So the necessary policy mix is one which
% %   increases the interest rate and leaves output unchanged. This
% %   requires an expansionary fiscal policy, which increases the interest
% %   rate but in the short run increases output, coupled with a
% %   contractionary monetary policy to keep output at its long run
% %   equilibrium level. The IS curve shifts right and the LM curve left
% %   in such a way that it intersects the new IS curve where the latter
% %   crosses the MRLE.  Since output is unchanged, the AD curve does not
% %   shift and the price level is unchanged both in the short and long
% %   run.

% %   \begin{enumerate}
% %   \item [a)]Expansionary fiscal and monetary policy. False, it would
% %     boost output in the short run.

% %   \item[b)] Contractionary fiscal policy and expansionary monetary
% %     policy. False

% %   \item[c)] None of the above. True
% %   \end{enumerate}

% % \item Distinguish between expectations about the \emph{current }price
% %   \emph{level} and expectations about inflations (the \emph{rate of
% %     change }of the price level between today and tomorrow). Derive
% %   Fisher equation has a no arbitrage condition between real and
% %   nominal assets. Tell them that the real interest rate matter for
% %   fixed investment decisions (IS) but the nominal one is the
% %   opportunity cost of holding money (LM). So changes in inflationary
% %   expectations shift the LM curve alone in the output/\emph{real
% %   }interest rate space. Instead changes in expectations about the
% %   price \emph{level} shift the SRAS alone. LM shifts down (money
% %   demand falls at unchanged $r$ as $i$ increases). So at unchanged
% %   real interest rate and prices output needs to increase to
% %   reestablish money market equilibrium. AD shifts right. \textbf{Do
% %     the (long) long run first. }

% %   \begin{enumerate}
% %   \item [a)]Real interest rate decreases. Prices increases but, as
% %     workers have incorrect \emph{price level }expectations, less than
% %     in the long run. So the LM curve does not shift all the way back
% %     and the real interest rate has to fall to clear the money market.
% %     The fall in the real interest rate boosts investment and output.

% %   \item[b)] Real interest rate is unaffected as neither the IS nor the
% %     MRLE\ have been affected (classical dichotomy holds and expected
% %     inflation is a nominal variable). The LM curve shifts all the way
% %     back. The nominal interest rate increases one for one. Output is
% %     unchanged and the price level needs to increase to bring the
% %     supply of real balances in line with lower demand. Note the
% %     crucial role of price adjustment to clear the money market without
% %     changes in the real interest rate.
% %   \end{enumerate}
% % \end{enumerate}
\end{document}