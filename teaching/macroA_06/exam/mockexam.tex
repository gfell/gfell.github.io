%%% Local Variables: 
%%% mode: latex
%%% TeX-master: t
%%% End: 
\documentclass[12pt,a4paper]{article}
\usepackage{fullpage}
\usepackage{amsmath}
\usepackage{amssymb}
\usepackage{setspace}
%\usepackage{times}
\setlength{\textheight}{24.5cm}
\setlength{\textwidth}{16cm}
\begin{document}
\doublespacing 

\pagestyle{empty}

Time allowed:  2 hours + 15 minutes reading time
\vspace{.7in}

\singlespacing
Answer ONE question from Section A and TWO questions from Section B.


\vfill
\textbf {Do not turn over until you are told to do so by the Invigilator.}
\\
\newpage
\vspace*{-.8in}
\begin{center}
  2
\end{center}


\vspace{.3in}

\underline {Section A}

\begin{enumerate}
\item (60 points) Assume an economy in which there are two activities:
  production of ideas or knowledge, $A$, and production of a final and
  intermediate good $Y$. Output is produced using capital $K$, labor $L$
  and knowledge $A$ according to the technology
  \begin{equation}
    Y=K^{\alpha }(A(1-a_{L})L)^{1-\alpha },
  \end{equation}
  where $0<\alpha <1$ and $a_{L}$ is the constant proportion of
  workers employed in the good sector. The exogenous rate of growth of
  the total labour force is $n.$ In each period, a constant fraction
  $s$ of output $Y$ is invested in new machines and the depreciation rate
  of the existing stock of machines is zero.
  New knowledge is produced using researchers and existing ideas
  according to the technology
  \begin{equation}
    \dot{A}=\delta (a_{L}L)^{\lambda }A^{\varphi },
  \end{equation}
  where $0<\lambda <1$, $\varphi >0$.

  \begin{enumerate}
  \item Write down the equation describing the accumulation of capital
    per efficiency unit of labour.  
  \item Derive the steady state rates of growth of the stock of ideas
    $A$, capital per worker $K/L$ and income per capita $Y/L$ in this
    economy when $%
    \varphi <1$ and $n>0$. What happens to these rates of growth when
    $\varphi \ $tends to one?
  
  \item Suppose $s=0.12,$ $n=0.02,$ $\lambda=0.3$ and $\varphi=0.7$.
    Derive the steady state value of $Y/K$ for the economy in point
    (b).
  
  \item Derive the steady state rates of growth of the stock of ideas
    $A$, capital per worker $K/L$ and income per capita $Y/L$ in this
    economy when $%
    \varphi =1$ and $n=0$.
  
    \end{enumerate}

\item  (60 points) Suppose the aggregate supply and demand curves are given by 
\begin{eqnarray*}
\text{AS }y_{t} &=&\alpha \left( p_{t}-E_{t-1}p_{t}\right)  \\
\text{AD }y_{t} &=&m_{t}-p_{t}+v_{t}
\end{eqnarray*}
where $v_{t}=v_{t-2}+\epsilon_{t}$ and $\epsilon_{t}$ is a white noise
error with zero mean and variance $\sigma _{\epsilon}^{2}.$ The
expectation operator $E_{t-1}$ is conditional on all information
available to private agents up to the \emph{beginning} of period
$t-1$; i.e. two periods out of date.  On the other hand, the policy
maker observes shocks with just one period delay.  So at time $t$ the
policy maker cannot observe $v_{t}$ but does observe $v_{t-i}$ for all $i>0.$
\begin{enumerate}
\item Write down the equilibrium vector for this economy.
\item Obtain an expression for the equilbrium level of output as a
  function of the money stock $m_{t}$ and the shock $v_{t}.$
\item  What is the variance of output if the policymaker follows the
publicly known policy rule $m_{t}=\bar{m}$ with $\bar{m}$ constant?

\newpage
\begin{center}
\vspace*{-.8in}
\hspace*{-1cm}  3
\end{center}

\item Suppose instead the policymaker follows the publicly known
  linear policy rule $m_{t}=\bar{m}+\gamma_{0} v_{t-1}
  +\gamma_{1}v_{t-2}$ Calculate the values of $\gamma_{0} $ and
  $\gamma_{1}$ that minimize the variance of output.
\end{enumerate}

\item   (60 points) Consider the Ramsey model.
  Households maximize
  \[U_{0}=\int_{0}^{\infty}\frac{C_{t}^{1-\theta}}{1-\theta}L_{t}e^{-\rho
    t}dt,\] 
where $C_{t}$ denotes consumption per household member,
  $L_{t}=e^{nt}$ is the household size and $\rho>0$ is the subjective
  discount rate. There is just one household in the economy. Output is
  produced according to the production function
  $F(K_{t},A_{t}L_{t})=K_{t}^{\alpha}(A_{t}L_{t})^{1-\alpha}$ where $K_{t}$ is
  the aggregate capital stock and $A_{t}=e^{gt}$ the efficiency of
  labour. Capital does not depreciate.
  \begin{enumerate}
  \item Write down the Lagrangean for the household problem and derive
    the Euler equation.
  \item Impose general equilibrium and derive the equations that
    characterize the evolution of capital  and consumption per unit of
    effective labour.
  \item At time $t_{0}$ the economy has a level of capital per head
    above its steady state value. Use a phase diagram to derive the
    equilibrium path for capital and consumption per unit of effective
    labour from time $t_{0}$ onwards.
  \item Suppose the economy is initially in steady state equilibrium
    and that at time $t_{1}$ a demographic shock reduces $n$
    permanently. Use a phase diagram to explain how the shock affects
    the time path of capital per unit of effective labour from time
    $t_{1}$ onwards.  
\end{enumerate}
\end{enumerate}


\bigskip

\underline {Section B}
\begin{enumerate}
\item [4.]   (20 points) State the Lucas-Sargent-Wallace Policy Ineffectiveness
  Proposition? What does it imply about the usefulness of
  macroeconomic stabilization?
\item [5.] (20 points) Discuss how the interplay of the consumption
  smoothing and consumption tilting motives shapes the response of
  capital accumulation to a permanent total factor productivity shock in
  a Real Business Cycle model with fixed labour supply.
\item [6.] (20 points) What technological features are necessary to
  generate endogenous growth? Discuss formally in the context of Arrow
  learning-by-doing model.
\item [7.] (20 points) What is time inconsistency? What does it imply
  for the conduct of monetary policy?
\end{enumerate}
\begin{center}
  END OF PAPER
\end{center}

\end{document}
