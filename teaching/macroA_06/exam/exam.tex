%; whizzy document -advi advi 

\documentclass[12pt,a4paper]{article}
\usepackage{fullpage}
\usepackage{amsmath}
\usepackage{amssymb}
\usepackage{setspace}
%\setlength{\textheight}{24.5cm}
%\setlength{\textwidth}{16.5cm}
\begin{document}
\setcounter{page}{1}

\noindent\Large {\textbf{QUEEN MARY, UNIVERSITY OF LONDON }}
\newline 

\noindent \large {M.Sc.(Economics), M.Sc. Financial Economics\vspace{0.4cm} }

\noindent\textbf{\large Macroeconomics A - ECOM001}\vspace{0.4cm}

\noindent \large {Date: 25 May  2007, 10:00 p.m.

\noindent Duration: 2 hours and 15 minutes (This includes 15 minutes
reading time)}

\vspace{4cm}

\normalsize 
\em
\noindent  Answer one question from Section A and two
questions from Section B. 

\bigskip
\noindent You are not permitted to start reading this question paper
until instructed to do so by an invigilator.

\bigskip
\noindent Complete all rough workings in the answer
book(s) and cross through any work that is not to be assessed.

\bigskip
\noindent Calculators are permitted in this examination
provided they are not programmable. Please state the name and type
of calculator on your answer book.
\em

\vfill \noindent \copyright Queen Mary, University of London
2007.\newpage

\noindent \textbf{Section A} 


\begin{enumerate}
\item (60 points) Suppose an economy is described by the following equations 
\begin{align}
&\text{(AS)\hspace{0.29in}}y_{t}=p_{t}-E_{t-1}[p_{t}]+u_{t}\nonumber \\ 
&\text{(AD)\hspace{0.29in}}y_{t}=m_{t}-p_{t}+v_{t}\nonumber
\end{align}

where $y_{t}$, $p_{t}$ and $m_{t}$ are the logarithms of output, price
and the nominal money stock and $u_{t}$ and $v_{t}$ are independent
shocks with variances $\sigma _{u}^{2}$ and $\sigma _{v}^{2}$
respectively. $E_{t-1}$ denotes expectations conditional on private
agents' information set at time $t-1.$ Private agents do not observe
shocks and their expectation of them is zero. The policymaker does not
observe the contemporaneous realization of the shocks, but perfectly
observes past ones.

\begin{enumerate}
\item Write down the equilibrium vector for this economy.
\item Obtain an expression for the equilbrium level of output as a
  function of the money stock $m_{t}$ and the shocks $u_{t}$ and
  $v_{t}.$

\item  What is the variance of output if the policymaker follows the
publicly known policy rule $m_{t}=\bar{m}$ with $\bar{m}$ constant?

\item Can the government reduce the variability of output
  using a different policy rule if both $u_{t}$ and $v_{t}$ are serially
  uncorrelated? What if $%
  u_{t}$ is uncorrelated, but $v_{t}=\rho v_{t-1}+e_{t}$ where $e_{t}$
  is a serially uncorrelated error with zero mean and variance $\sigma
  _{e}^{2}<\sigma _{v}^{2}$? Do your results contradict Sargent and
  Wallace's Policy Ineffectiveness Proposition?
\end{enumerate}
\item (60 points) Consider the Ramsey-Cass-Koopmans model.  Households
  maximize
  \[U_{0}=\int_{0}^{\infty}\frac{C_{t}^{1-\theta}-1}{1-\theta}L_{t}e^{-\rho
    t}dt,\] 
where $C_{t}$ denotes consumption per household member,
  $L_{t}=e^{nt}$ is the household size and $\theta$ and $\rho$ are
  positive parameters. There is just one household in the economy. Output is
  produced according to the production function
  $F(K_{t},A_{t}L_{t})=K_{t}^{0.5}(A_{t}L_{t})^{0.5}$ where $K_{t}$ is
  the aggregate capital stock and $A_{t}=e^{gt}$ the efficiency of
  labour. Capital does not depreciate.
  \begin{enumerate}
  \item Write down the Lagrangean for the household problem and derive
    the Euler equation.
  \item Impose general equilibrium and derive the equations that
    characterize the evolution of capital  and consumption per unit of
    effective labour.
  \item Assume $\rho=0.04$, $\theta=1/3$ and $n=g=0.03.$ Derive the
    steady state values of capital and consumption per unit of effective
    labour.

\vfill \hfill continues on the next page\newpage

  \item Suppose the economy is initially in steady state equilibrium
    and that at time $t_{0}$ a shock to preferences increases $\theta$
    permanently. Use a phase diagram to explain how the shock affects
    the time path of capital per unit of effective labour from time
    $t_{0}$ onwards.  
\end{enumerate}



\item (60 points) Consider an economy in which the aggregate supply is
  given by
\[
y_{t}=\pi _{t}-E_{t-1}[\pi _{t}]+v_{t} 
\]
where $y_{t}$ is the logarithm of output, $\pi _{t}$ is inflation and
$v_{t}$ is a serially uncorrelated shock with zero mean and variance
$\sigma ^{2}.$ $E_{t-1}$ denotes expectations conditional on private
agents' information set at time $t-1.$ The shock is observed by the
policymaker but not by the private sector. The government welfare
function is given by
\[
W^{g}=-\pi _{t}^{2}-(y_{t}-\gamma )^{2}, 
\]
where $\gamma>0$ is the target level of output.  The order of events
is the following.  First, at time $t-1$ the government decides whether
to commit on a specific rule or not. Second, the private sector forms
its expectations, also at time $t-1$. Third, at time $t$ the shock
$v_{t}$ is observed by the policymaker who then chooses the rate of
inflation.

\begin{enumerate}
\item  Write down the equilibrium vector for this economy.
\item Suppose the government can credibly precommit to the linear rule $\pi
_{t}=\alpha $, where $\alpha $ is a constant parameter. What is the optimal
value of $\alpha $ and the associated values of the inflation rate, output
and the expected value of the loss function?\ What are the costs and
benefits of this rule, if any?

\item Suppose that, instead, the government decides to delegate
  monetary policy to an independent central banker with the same
  preferences and information set as the government but whose income
  is a linear function of the rate of inflation (i.e. her pay is
  performance related). The central banker's welfare function is then
\[
W^{b}=-\pi _{t}^{2}-(y_{t}-\gamma )^{2}+\beta-2\delta\pi _{t} 
\]
where $\beta-2\delta \pi _{t}$ is the banker's pay. What are the
associated values of inflation and output?

\item What is the optimal value of $\delta$ from the government's
  point of view and the expected value of the government loss function
  given the optimal $\delta$? Compare the result with that in (b).
  What is the economic intuition behind your findings?
\end{enumerate}
\end{enumerate}
\bigskip
\vfill \hfill continues on the next page\newpage
\textbf {Section B}
\begin{enumerate}
\item[4.] (20 points) Using the Solow growth model, discuss dynamic
  inefficiency. What does dynamic inefficiency imply about the response
  of steady state aggregate consumption to an increase in the aggregate
  saving rate?
\item [5.] (20 points) Explain under what conditions a government is
  solvent.  Discuss whether government solvency is necessary for
  Ricardian equivalence to hold.
\end{enumerate}
\begin{enumerate}
\item [6.] (20 points) Discuss the difference between weak and strong
  scale effects in endogenous growth models. Highlight relevant
  empirical evidence supporting or questioning either type of effect.
\item [7.] (20 points) How does Real Business Cycle theory explain
  output fluctuations? Discuss its success or lack thereof in accounting
  for the observed persistence of output.
\end{enumerate}


\hfill \textbf{End of examination}

\hfill \textbf{Dr.\ G.\ Fella}


\end{document}
