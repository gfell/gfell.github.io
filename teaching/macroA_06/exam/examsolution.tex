\documentclass[12pt,a4]{article}
\usepackage{fullpage}
\usepackage{amsmath}
\usepackage{amssymb}
\usepackage{setspace}
\begin{document}

\vspace{.5in}
\textbf{Solution to the Macro A exam paper }

\bigskip
\textbf {Section A}

\begin{enumerate}
\item
\begin{enumerate}
\item The equilibrium vector is
  $\left[p_{t},E_{t-1}p_{t},y_{t},E_{t-1}y_{t}\right]$ such that (AS)
  and (AD) are satisfied and expectations are mathematical
  expectations conditional on the agents' information set.
\item It is
  \begin{align}
    \label{eq:1}
    p_{t}-E_{t-1}p_{t}=\frac{1}{2}\left[m_{t}-E_{t-1}m_{t}+v_{t}-u_{t}\right]\\
y_{t}=\frac{1}{2}\left[m_{t}-E_{t-1}m_{t}+v_{t}+u_{t}\right]
  \end{align}

\item  From (2) it is $(\sigma^{2}_{u}+\sigma^{2}_{v})/4.$ 

\item  No if the shocks are uncorrelated, as the government has no
  informational advantage over private agents. If $v_{t}$ is
  autocorrelated the government has an informational advantage has it
  observes $v_{t-1}$ while private agents do not. Since private agents can only
  forecast the shock-independent component of the monetary rule the
  government can reduce the output variance by following the feedback rule
    $m_{t}=\overline{m}-\rho v_{t-1}.$  
This implies
\begin{equation}
  \label{eq:4}
  y_{t}=\frac{1}{2}\left[\bar{m}-\rho v_{t-1}-\bar{m}+v_{t}+u_{t}\right]=\frac{1}{2}\left[e_{t}+u_{t}\right].
\end{equation}
The associated output variance is $(\sigma^{2}_{u}+\sigma^{2}_{e})/4.$

The result does not contradict
  the PIP which assumes symmetric information and flexible prices.
\end{enumerate}
\item Lower case letters denote variables in efficiency units of
  labour. 
  \begin{enumerate}
  \item The associated Lagrangean is
\begin{equation}
  \label{eq:16}
 \mathcal{L}=\int^{\infty}_{0}\frac{c_{t}^{1-\theta}}{1-\theta}e^{-[\rho-(1-\theta)g-n]t}dt+\lambda \left[b_{0}+\int^{\infty}_{0}(w_{t}-c_{t})e^{-R_{t}+(n+g)t}dt\right].   
\end{equation}
Students were not required to be able to derive the intertemporal
budget constraint.

The  first order condition for optimal consumption is 
\begin{equation}
  \label{eq:18}
  c_{t}^{-\theta}=e^{-[R_{t}-(\rho+\theta g)t]}.
\end{equation}

Taking logs and time derivatives the FOC can be rewritten as the Euler
equation
\begin{equation}
  \label{eq:19}
  \frac{\dot{c_{t}}}{c_{t}}=\frac{r_{t}-\rho-\theta g}{\theta}.
\end{equation}

\item General equilibrium requires $b_{t}=k_{t}$ and factor prices to
  equal the factor's marginal products. The two differential equations
  that characterize the evolution of $c$ and $k$ are
\begin{equation}
  \label{eq:23}
  \dot{k_{t}}=f(k_{t})-c_{t}-(n+g)k_{t}.
\end{equation}
and 
\begin{equation}
  \label{eq:25}
\frac{\dot{c}}{c}=\frac{f'(k_{t})-\rho-\theta g}{\theta}.  
\end{equation} 
To these one must add the solvency constraint evaluated at equilibrium\\
$\lim_{t\to \infty}k_{t}e^{-R(t)+(n+g)t}\geq0.$   
  \item From the Euler equation we obtain that in steady state it is
    $0.5k^{-.5}=\rho+\theta g=0.05$ or $k=100.$ The goods market
    clearing  condition implies $c=\sqrt{100}-.06\times100=4.$
  \item The vertical $\dot c=0$ locus shifts left while the $\dot k=0$
    locus is unaffected. $c_{t_{0}}$ jumps
    up onto the new saddle path for consumption to fall on the
    transition to the new steady state ($f\left(k\right)<\rho+\theta
    g$ on the transition path). Capital and consumption fall
    along the transition path as they converge to lower values in the
    new steady state.
\end{enumerate}
\item The FOC for the general case can be derived by maximizing the central
banker's welfare function subject to the SRAS constraint. One obtains
\begin{equation}
  \label{eq:2}
  \pi_{t}=\frac{1}{2}\left[E_{t-1}\pi_{t}+\gamma-d-v_{t}\right].
\end{equation}
This implies
\begin{equation}
  \label{eq:3}
  E_{t-1}\pi_{t}=\gamma-d
\end{equation}
in the absence of commitment. $d=0$ in part (b) of the question. 
\begin{enumerate}
\item $\left[p_{t},E_{t-1}p_{t},y_{t},E_{t-1}y_{t}\right]$ such that
  the aggregate supply curve is satisfied, the policymaker FOC is
  satisfied and expectations are ``rational''.
\item Since the commitment is credible it is
  $\pi_{t}=E_{t-1}\pi_{t}=\alpha$ and $y_{t}=v_{t}.$ Replacing in the
  government objective function and taking expectations yields
  $W=-\alpha^{2}-\sigma^{2}-\gamma^{2}$ which is minimized
  for $\alpha=0.$ Too rigid and suboptimal since government dislikes
  output fluctuations.

\item  From  (\ref{eq:2}) the equilibrium value of inflation is
  $\pi_{t}=\gamma-d-v_{t}/2.$ Equilibrium output is $y_{t}=v_{t}/2.$

\item  It is optimal for the government to choose $d=\gamma$ which
  sets the inflation bias to zero. The associated value of government
  welfare is $W^{g}=-\sigma^{2}/2-\gamma^{2}.$ Relative to case (b) the optimal
  contract induces the central bank to stabilize output fluctuations
  which is desirable given that output enters government welfare quadratically.
\end{enumerate}
\end{enumerate}
\newpage
\textbf{Section B}
\begin{enumerate}
\item [4.] With an exogenous saving rate the steady state stock of capital in
  efficiency units satisfies $f(k^{*})/k^{*}=(\delta+n+g)/s.$ It is
  possible that $k^{*}$ exceeds the golden rule value $k_{GR}$ which
  maximizes steady state consumption and satisfies
  $f'(k_{GR})=\delta+n+g.$ In such a case the economy is dynamically
  inefficient.  Increasing the saving rate reduces steady state
  aggregate consumption.
\item [5.] Solvency requires the stock of government debt not to exceed the
  present value of future surpluses. It is normally assumed that the
  solvency constraint holds as an equality. If this is the case, for a
  given path of government expenditure, the present value of taxes is
  fully pinned down. Therefore current tax cuts imply future tax
  increases of equal present value. It is clearly a necessary
  condition for Ricardian equivalence to hold.
\item [6.] Weak scale effects are associated with increasing returns to
  capital and labour but decreasing returns to reproducible factors
  (capital and the stock of knowledge). They imply that policy does
  not affect the steady state rate of growth of the economy (no level
  effects) though the latter depends on the rate of growth of the population.\\
  Strong scale effects are associated with constant returns to
  reproducible factor (e.g $AK$ model) and imply that the growth rate
  of the economy depends on the size of the population and policy
  variables such as saving rates.\\
  From an empirical point of view both cannot explain productivity
  differences in the absence of cross-country barriers to
  technological adoption (all countries would have access to the same
  technology).\\
  As theories of what determines the world (frontier) level of TFP
  weak scale effects are more plausible since otherwise growth rates
  should respond to changes in saving rates or population sizes which
  we do not observe.


\item [7.] TFP shocks are the main source of fluctuations. Since output
  fluctuations are associated with large employment fluctuations and
  little fluctuations in labour productivity, the model requires a
  large intertemporal elasticity of labour substitution (one way or another).\\
  With the measured capital share equal to 0.3 in the data the model
  generates little persistence in output over and above the
  persistence of TFP shocks. Also little amplification. TFP shocks
  must be highly variable and persistent.

\end{enumerate}
\end{document}
