%; whizzy document -advi advi 

\documentclass[a4paper,12pt]{article}
\usepackage{latexsym}
\usepackage{fullpage}
\usepackage{amsmath}
\usepackage{amssymb}
\usepackage{graphicx}

\begin{document}

\noindent\Large {\textbf{QUEEN MARY, UNIVERSITY OF LONDON }}
\newline 

\noindent \large {M.Sc.(Economics), M.Sc. Financial Economics\vspace{0.4cm} }

\noindent\textbf{\large Macroeconomics A - ECOM001} (resit/first sit)\vspace{0.4cm}

\noindent \large {Date: 25 May  2007, 10:00 p.m.

\noindent Duration: 2 hours and 15 minutes (This includes 15 minutes
reading time)}

\vspace{4cm}

\normalsize 
\em
\noindent  Answer one question from Section A and two
questions from Section B. 

\bigskip
\noindent You are not permitted to start reading this question paper
until instructed to do so by an invigilator.

\bigskip
\noindent Complete all rough workings in the answer
book(s) and cross through any work that is not to be assessed.

\bigskip
\noindent Calculators are permitted in this examination
provided they are not programmable. Please state the name and type
of calculator on your answer book.
\em

\vfill \noindent \copyright Queen Mary, University of London
2007.\newpage

\noindent \textbf{Section A} 

\begin{enumerate}
%
%
\item (50 points) Consider the Solow growth model with constant saving
  rate $s$, population growth rate $n$ and no depreciation rate. Let the
  production function be

\begin{center}
$F(K,L)=K+K^{\alpha}L^{1-\alpha}$ ,

\end{center}
with $0<\alpha<1$.
\begin{enumerate}
	\item Is the economy characterised by constant returns to scale?
	\item Let $k$ be capital per capita. Write the equation describing capital accumulation in
   per capita terms.
   \item Derive restrictions on parameters which ensure $k$ is finite along the balanced
   growth path.
   \item Find the share of labour along the balanced growth path as a function of $s$ and $n$.
	 \item Discuss whether the model balanced growth path is consistent with Kaldor's
   stylised facts?
\end{enumerate}
%
\item (50 points) Consider the following version of the Human Capital
  model. The model is defined by the set of equations

$Y=K^{\alpha}L^{1-\alpha}$,

$\dot K=s_{K}Y$,

   $\dot H=s_{H}Y$,
with $ 0<\alpha <1$ , $0<s_{K}<1$ , $0<s_{H}<1$.
\begin{enumerate}
	\item Discuss the economic interpretation of the different parts of the model.
	\item Show that $K/H$ converges to a balanced growth path level.
	\item Compute the growth rates of $K/H$ and $Y$ along the balanced growth path.
  \item How does the growth rate of output depends on the investment rates $s_{H}$ and $s_{K}$ ?
  \item Assuming that $K/H$ is initially above its balanced growth value, how does the
   initial growth rate of $Y$ compare to the growth rate of $Y$ along the balanced growth
   path?
\vfill
\begin{flushright}
continued on next page...
\end{flushright}
\end{enumerate}
\end{enumerate}
\newpage
\textbf{Section B}

\begin{enumerate}
\item [3.] (25 points) In this question you will review some issues
  partaining to the theory of Real Business Cycles (RBC).

\begin{enumerate}
\item Describe the stylised empirical facts concerning the economic fluctuations. In
   particular compare the volatility of GDP to the volatility of investment, of
   government consumption and private consumption of non-durable goods.
\item Compare the 1929 Great Depression and the 1974 recession. In particular, discuss
   the movement of unemployment, labour productivity, investment, consumption
   and the price level in the two episodes.
\item According to the Real Business Cycles approach what is the mechanism
   producing economic fluctuations? Is this explanation plausible? Discuss.
\item What are the main other assumptions of the RBC Theory? To which extent do
   fluctuations represent optimal responses to shocks?
\item Are fluctuations in the price level an exogenous shocks? Describe the two major
   post-war oil shocks.
\item What is the effect of changes in the quantity of money on output? Do changes in
   the price level affect output? Is unemployment voluntary or involuntary?
\item Describe the notion of ``endogenous fluctuations''. Discuss an economic model in
   which endogenous fluctuations may occur. Are the conditions ensuring the
   occurrence of endogenous fluctuations realistic?
\end{enumerate}   
%
\item [4.] (25 points) Consider the two-period, pure exchange,
  overlapping generation model Assume that the individual utility is
  $u(x,y)=\ln(x)+\delta\ln(y)$ and that the initial endowments are
  $(e^{y},e^{o})=(4,1)$.
\begin{enumerate}
\item Briefly, describe the stylised empirical facts concerning the economic fluctuations.
\item Give the main structure of the overlapping generations pure exchange model
   (without production).
\item Relate the real interest rate between time $t$ and $t+1$ to the present prices in period
   $t$ and $t+1$.
\item Write and solve the consumer maximization program (you can use interest rates or
   present prices)
\item Draw the offer curve. Draw the dynamic curve $\Gamma$. How is the latter obtained?
\item State the equation concerning the evolution of interest rates (or present prices).
\item What are the steady states? Are these stable? What is the value of real savings at
   these?
 \item Draw some indifference curves and an associated offer curve which
   is consistent with endogenous fluctuations.
\vfill
\begin{flushright}
continued on next page...
\end{flushright}
\end{enumerate}
%
\item [5.] (25 points) Consider a pure exchange intertemporal model that
  extends over two periods. There is one perishable good per period.
  There are two consumers with utility function $u(c_{1}, c_{2})=\log
  c_{1}+\log c_{2}$ where $c_{1}$ and $c_{2}$ denote consumption in
  period 1 and 2, respectively. The initial endowments of the first
  consumer are $\omega_1(1),\omega_1(2)=(2,1)$ while for the second
  these are $\omega_2(1),\omega_2(2)=(1,2)$.
\begin{enumerate}
	\item Are total resources and preferences independent of time?
  \item Do you expect individual consumption to be constant or vary across periods?
\item For consumer 1, draw his indifference curve passing through the point
representing his initial endowments. Find its analytical expression.
\item Let $r$ be the interest rate. What is the individual demand for consumer 1?
\item Can there be an equilibrium with zero interest rate?
\item Is there an equilibrium with strictly positive interest rate?
\item Explain why the model depicts a situation in which endogenous fluctuations
  are impossible.
\end{enumerate}
   
\end{enumerate}


\begin{flushright}
\textbf{End of Examination} 

\textbf{Dr. G. Fella}
\vspace{1em}
\end{flushright}
\end{document}
